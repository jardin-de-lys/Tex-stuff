\setlength{\chnumsep}{10em}
\chapter{Kinematics}


\begin{overview}
    Kinematics is the study of motion without any concern to its cause. It introduces us with the 
    basic quantities that will used throughout physics. While there's not that much physical content in this 
    topic, it stands as quite an important topic in mechanics. Once you've solved the dynamic equations, all 
    that's left is kinematics.
        \end{overview}

\section{The Physical Quantities}

\subsection{Frames and Particles}

Motion of any object is considered to relative to an observer. The observer defines a
particular co-ordinate system called the reference frame.

\begin{definition}
    A \vocab{Reference Frame} is a co-ordinate frame which can be moving or rotating, with respect to 
    which we measure the physical quantities.
\end{definition}

There are some things that depend on our choice of reference frame, and some that
don't. We'll have a look at using reference frames to deal with some questions
later. 

\begin{marginfigure}
    \vspace{-5em}
    \scalebox{2}{\incfig{reference-frame}}
    \caption{A cartesian reference frame}
\end{marginfigure}


Typically, we either use cartesian or curvilinear co-ordinates (spherical, cylindrical). 
We'll have a look at the 2d versions of both in this chapter, especially plane polar-coordinates 
(2d versions of cylindrical co-ordinates).

We usually deal with either point particles, or rigid bodies.  You might think a point 
particle is not really a physically meaningful thing. However, we can
approximate bodies, when their size is \emph{not} meaningful to their
motion, as particles to a very good precision. 

\begin{definition}
    A \vocab{Point Particle} is a particle whose size is negligible in the study of its motion.
\end{definition}

We'll later discuss extended bodies, and in particular rigid ones in mechanics, but initially this idea 
of point particles will carry us far. 

\subsection{Position and Displacement}

To describe the position of the particle, we use a position vector. A position vector is an example of a 
\emph{vector field}, about which we'll have much to say in electromagnetism and gravitation.

\begin{definition}
    The \emph{position vector}, $\vec{r}(x, y, z)$, of the particle refers to the vector drawn from the origin to the particle.
\end{definition}


In \cref{fig: position}, the position vector of $A$ is $\vec{r}_A = x\ihat + y\jhat + z\khat$, where 
$\ihat$ is the unit vector along the $x$ axis, $\jhat$ the unit vector along the $y$ axis and $\khat$ is 
the one along the $z$ axis.

As you may notice, this is dependent on our origin. The position vector is not quite
a true vector, it's more so a relative quantity. Anyway, it is still plenty useful,
if we just remember to adjust it when switching to a different reference frame.

\begin{marginfigure}
    \hspace{-1.2em}\scalebox{2.5}{\incfig{positionvec}}
    \caption{A position vector $\vec{r}_A = (x, y, z)$} 
    \label{fig: position}
\end{marginfigure}

The other thing of importance is \emph{Displacement}. 
The \index{Displacement} \emph{Displacement}, $\vec{s}_{AB}$ from $A$ to $B$, is $\vec{s}_{AB} \equiv \vec{r}_B - \vec{r}_A$.

We need two points to define the displacement. Also, unlike position, displacement 
is not changed by using a rotated or translated reference frame. 

Note that displacement is not the property of the path, which is the \emph{distance}. For a path 
of the form $y(x)$, to calculate the distance between $A$ and $B$, we need to calculate $\dd{\ell} = 
\sqrt{\dd{x}^2 + \dd{y}^2} = \sqrt{(\dv*{y}{x})^2 + 1}\dd{x}$ which is the \vocab{arc length}.
$\dd{\ell}$ is just the infinitesmal length of the path, which by pythagoras is equal to $\sqrt{\dd{y}^2 + \dd{x}^2}$.  

Doing this integral, we get,
\begin{equation}
    \ell = \int_A^B \sqrt{(y'(x))^2 + 1} \dd{x}.
\end{equation}

This is the distance. For euclidian geometry, it is also obvious that since displacement is the line joining 
$A$ and $B$, the magnitude of displacement is lower than the length of any path joining $A$ and $B$. Thus,
$\norm{\vec{s}} \le \ell$. 

\subsection{Velocity And Acceleration}

And now we can start by considering the quantities we have heard about frequently
in our daily life, velocity and acceleration. Velocity refers to how the position
vector changes with time, while acceleration refers to how velocity changes with time.
So, acceleration is the second derivative of position. 

You might wonder why we don't consider higher derivatives, and the answer to that lies in the 
fact that newton's laws give us a second order differential equation, so the only 
quantities of concern are till the second derivative of position.

Let's now formally define these quantities.

\begin{definition}
    \textit{Velocity} is defined as the time rate of change of position, i.e. $\vec{v} \equiv \dot{\vec{r}}$.

    \textit{Acceleration} is defined as the time rate of change of velocity, i.e. $\vec{a} \equiv \dot{\vec{v}} = \ddot{\vec{r}}$. 
\end{definition}

\sidenote{$\dot{x}$ refers to the derivative of $x$ with respect to time. $\ddot{x}$ refers to the second derivative of $x$ with respect 
to time. `$\equiv$' stands for defined as.}

\begin{remark}
    Some quick remarks. Position is a function of time, and to find said function
    is the aim of mechanics. We call the function the \textit{trajectory} of the 
    particle. There are some constraints on the function, for example it must have 
    a continuous first, second derivative. 

    Similar, velocity and accelerations are also functions of time. Notably as you can
    see from their definitions, they're defined at a particular time, the derivative
    evaluated at some time, $t$. This is different from things like \textit{average velocity}, which is a property of motion in a finite time interval, explicitly,
    \begin{equation}
        \vec{v}_{\text{avg}} = \vec{\bar{v}} \equiv \frac{\vec{r}(t_2) - \vec{r}(t_1)}{t_2 - t_1}.
    \end{equation}

    And in the limit $t_2 \to t_1$, this becomes the velocity. The case for acceleration
    is the same.

    Speed at a time $t$ is the magnitude of velocity at that time. It is always greater
    than or equal to $0$. In the case of average speed, it is defined as the \textit{distance} the particle travels over the
    time interval in which it travels that distance. 

    When the time interval goes to $0$, the distance becomes the displacement, or atleast the magnitude of 
    it. Thus, speed is the magnitude of velocity at that point. 
    Also, don't confuse between $\norm{\dv*{\vec{r}}{t}}$ and $\dv*{\norm{\vec{r}}}{t}$, the first is speed, 
    the other is a quantity not at all useful.
\end{remark}

The velocity is a map,
$$
\vec{v}\colon \mathbb{R} \to \mathbb{R}^3
$$
similar to position, which maps time to its three components,
$$
t \mapsto (v_x(t), v_y(t), v_z(t)).
$$

Based on the definition of acceleration,
\begin{equation}
    \int \dd{\vec{v}} = \int \vec{a}\dd{t} \implies \vec{v} = \vec{v}_0 + \int \vec{a} \dd{t}.        
\end{equation}

Similarly,
\begin{equation}
    \vec{r} = \vec{r}_0 + \int \vec{v} \dd{t}.
\end{equation}

If acceleration is a function of position, which it often is, we can use a neat trick, at least when 
every thing is one dimension,
$$
a = \dv{v}{t} = \dv{v}{t} \dv{x}{x} = \dv{v}{x} \dv{x}{t} \implies a = v \dv{v}{x}.
$$
For constant acceleration, this gives us, $v^2 = v_0^2 + 2a\Delta x$, which is also the statement of work-energy 
theorem, which we'll encounter later. 

You can generalise this to any function, $f$,
\begin{equation}
f''(x) = f'\frac{d\hspace{-0.35em}f'}{d\hspace{-0.35em}f}.
\end{equation}

\begin{example}
    Consider the motion of an object dropped from a height $h$, under 
    drag. The acceleration due to gravity is $\vec{g} = -g \hat{\vec{j}}$ and due to drag is of the form $-\alpha v \hat{\vec{v}}$. This is 
    called linear drag. Since the object is falling down, $\hat{\vec{v}} = -\hat{\vec{j}}$.

    Based on the definition of acceleration, 
    $$
    \dv{v}{t} = - g + \alpha v \iff \int_0^{-v_f} \frac{\dd{v}}{\alpha v - g} = \int_0^{t} \dd{t'}
    $$
    where we set the final velocity to $-v_f$ since its negative, and rename $t$ to $t'$ since we'll use it as our dummy variable, and $t$ as the limit of integration.

    This can be solved using a simple u-sub, and is left as an exercise
    to the reader. The final solution is,
    $$v_f = \frac{g}{\alpha} (e^{\alpha t} - 1).$$
\end{example}

\begin{example}
    A car is at distance $d$ from a boy. It starts accelerating at constant acceleration $a$.  
    What is the minimum velocity that the boy should have to catch up with the car?

\begin{soln}
    Consider separation of boy and car, $\Delta s$. Integrating twice for constant acceleration, we have  
    $s_b = vt$, $s_c = \dfrac{1}{2}at^2$. Thus,
    \[
    \Delta s = d + \frac{1}{2}at^2 - vt
    \]
    So when they meet, we have a quadratic in $t$,
    \[
    at^2 - 2vt + 2d = 0
    \]
    From an inspection of the co-efficients of the $t$, $t^2$ terms and the constant, we see that if a real solution to this exists, 
    it must be positive (if this may not be apparent, recall Vieta's relation and note $a, v, d$ are all positive).

    So they must always meet if this has a real solution. Therefore, $b^2 - 4ac \geq 0$ for the equation
    $at^2 + bt + c$. Solving for this, we have:
    \[
    v \geq \sqrt{2ad}
    \]

    Alternatively, moving into the frame of the boy (we discuss the later, so you can skip it for now), 
    $v_f^2 = v^2 - 2ad$. Since $v_f^2 \ge 0$, $v^2 \ge 2ad$.
\end{soln}
\end{example}

\begin{example}
    A body is dropped at $t = 0$, after time $t = t_0$, another body is thrown downwards with velocity 
    $u$. Assuming first body reaches ground first, plot graph of separation.
    \begin{soln}
        At instant $t_0$, displacement of first particle = ${gt_0^2}/{2}$. Note that here we set up 
        co-ordinates such that positive $y$ is downwards from point of drop. 
        The displacement of first body at time $t$ after $t_0$ but before \textit{reaching ground} is
        \[
        s_1 = \frac{1}{2}gt_0^2 + \frac{1}{2}g(t - t_0)^2
        \]
        While for second body is,
        \[
        s_2 = ut + \frac{1}{2}g(t - t_0)^2 
        \]
        Thus,
        \[
        s_{1,2} = \frac{1}{2}g(t_0)^2 + t(g t_0 - u) 
        \]
        However, after first body reaches ground,
        \[
        s_{1,2} = ut + \frac{1}{2}g(t)^2 
        \]
        which is a parabola. The overall graph is:
        \begin{center}
            \begin{tikzpicture}[scale=0.7]
            \draw[->] (0,0) -- (5,0) node[right] {$t$};
            \draw[->] (0,0) -- (0,4) node[above] {$s$};
            
            % Draw linear part of the graph
            \draw[hkgreen, thick, domain=0:2] plot(\x, {0.8*\x + 1});
            
            % Draw parabolic part of the graph
            \draw[hkgreen, thick, domain=2:3.559026084] plot(\x, {3.8 - 0.3*\x^2});
            
            % Dots to mark transition
            \filldraw[slategreen] (2,2.6) circle (1pt);
            \end{tikzpicture}
        \end{center}
    
    \end{soln}
\end{example}

\section{Projectile Motion}

    Projectile Motion is a very famous setup in physics, and especially kinematic
    problems. Though the setup is not extremely physically interesting, there's a 
    number of things you can learn from it. We'll cover a lot of cool tricks through
    problems, but let's first explore what this setup even is.

    When an object, called a \textit{projectile}, is thrown with some velocity $v$ 
    making some angle with the horizontal, $\theta$, from co-ordinates $(X, Y)$,
    under the effect of gravity, the motion is called projectile motion.

    There are a no. of things in this setup that can teach you a fair bit. For one,
    this is our first actual two dimensional problem. We have two free co-ordinates,
    $x$ and $y$. First of all, Let's choose our co-ordinate system so that $X = 0$. 
    We won't choose $Y = 0$ because that corresponds to a real thing, namely the ground.
    It is more convenient to just leave it be.

    First, let's discuss the special case $Y = 0$, that is, projectile is released from
    ground. This setup should be enough for you to be able to generalise it to $Y = h$.
    Anyway, gravity acts in the negative y direction, $\vec{g} = -g \jhat$, where $g = 9.81$ m/s. How do we proceed now? Well, we know $\vec{a}(t)$ so we should probably try to find $\vec{v}(t)$. Let's try that,
    $$
    \dot{\vec{v}} = \vec{g} \implies \vec{v}(t) = \vec{v}_0 +\vec{g}t,
    $$
    Where we use $\vec{v}(0) = \vec{v}_0$, the initial velocity as the limit of integration.
    $$
    \dot{\vec{r}} = \vec{v}_0 + \vec{g}t \implies \vec{r} = \vec{r}_0 + \vec{v}_0t + \frac{1}{2} \vec{g}t^2.
    $$
    
    Since $\vec{r}_0 = (X, Y) = (0, 0) = \vec{0}$, we just have,
    $$
    \vec{r}(t) = \vec{v}_0t + \frac{1}{2} \vec{g}t^2.
    $$

    Now here's where the neat part begins. As a general rule of thumb, it's always
    harder to deal with vectors than it is with scalars, so we will see various
    methods of trying to convert a vector problem into a scalar problem, and then
    re-converting if needed. In this case, we make use of a very important, but almost
    trivial looking identity,
    $$\text{if } a\ihat + b\jhat = c\ihat + d\jhat, \text{then } a = c, b = d.$$

    How do we prove this? Well lets move things around a little, 
    $$
    (a-c)\ihat = (d-b)\jhat
    $$
    Oh, but this cannot be true since $\ihat, \jhat$ are the basis for $\mathbb{R}^2$ so they're linearly independent. Thus, either $a = b$, or $c = d$. Now you can complete the proof by using the uniqueness of $0$. A fancier way to do this is just 
    dot both sides with $\ihat$ and then $\jhat$, I'll leave that proof to you.

    Now, if $\vec{v}_0$ makes an angle $\theta$ with horizontal, 
    $$
    \vec{v}_0 = v_0\cos\theta \ihat + v_0\sin\theta \jhat,
    $$
    And plugging this into the vector equation for $\vec{v}(t)$, and using $\vec{v}(t) = v_x \ihat + v_y \jhat$ we get,
    \begin{equation}
        v_x = v_0\cos\theta, \quad v_y = v_0\sin\theta - gt        
    \end{equation}    

    And using the vector equation for $\vec{r}(t) = x\ihat + y\jhat$,
    \begin{equation}
        x = v_0t\cos\theta, \quad y = v_0t\sin\theta - \frac{1}{2}gt^2.        
    \end{equation}
    
    We have reduced the vector equations to scalar equations! Now this is very easy
    to deal with, you can derive almost everything by just bashing using these equations.
    That's of course not always the nicest idea, and we'll learn some tricks to deal with them. Also, you might have notice that for each vector equation, we get two scalar
    equations. We actually get three, since we live in 3d space, but the components of 
    the vector along $\khat$ are just $0$, so it doesn't matter.

    The last fundamental equation of importance is the \textit{trajectory} equation,
    how does the motion actually look in 2d space? That is, we need to find $y(x)$.
    You can do it by using $t = x/v_0\cos\theta$, and substituting this in the $y(t)$
    equation, which you should verify, gives you,
    \begin{equation}
        y(x) = x\tan\theta - \frac{x^2g}{2v_0^2\cos^2\theta}.        
    \end{equation}
    
    Now we can use this for a bunch of problems. Now that we know the general shape is a parabola,
    let's try to find some properties of it. As you can see, the parabola opens downards, so it
    achieves a maximum. What is that maximum? Well, you could try to find where $\dv*{y}{x}$ is 0,
    and then see what happens. 
    
    A nicer way is to realise that $v_y$ must be 0 at max height. If this were not so, 
    we could have moved to $H_{\text{max}} + \epsilon$ in time $t = \epsilon/v_y$. If the velocity were 
    negative at, and thus just before $H_{\text{max}}$, we could have never reached the max height.

    Using this, we get that,
    \begin{equation}
        H_{\text{max}} = \frac{v_{0y}^2}{2g} = \boxed{\frac{v_0^2\sin^2\theta}{2g}.}
    \end{equation}

    What about the range(the max value of $x$)? We have a maximum, since after $y = 0$ (after $t > 0$ that is), the projectile cannot
    move further, because it has encountered the ground. 
    
    Well, we can use the equation for $y$ to
    find $t$, or just note that the parabola is symmetric about $y = H_{\text{max}}$, so $t_{\text{total}} = t_H$ where $t_H$ is the time to reach $H_{\text{max}}$, or 
    equivalently, when $v_y = 0$, which you can find easily. All of them give the same time, which when plug into the equation for $x$, we get
    \begin{equation}
        \boxed{R = \frac{v_0^2\sin2\theta}{g}}.    
    \end{equation}

    Using this, we can also alternatively write the equation of trajectory as,
    \begin{equation}
        y = x\tan\theta\left(1 - \frac{x}{R}\right)
    \end{equation}

    What is the maximum value of $R$, the range, for a fixed velocity? Think about we can vary. Well, $\theta$ of course, and its easy to see the maximum occurs for $\theta = \pi/4$,
    so we \textit{optimise} the range, and say its maximum at $\theta = \pi/4$. 

    \begin{marginfigure}
        \vspace{10em}
        \hspace*{-4em}
        \scalebox{3.9}{\incfig{wedge-projectile}}
        \caption{Projectile grazing a wedge}
    \end{marginfigure}

    \begin{example}
        A projectile is thrown from one vertex of a wedge grazes the other vertex, and ends up on the third vertex. 
        It is thrown with an initial velocity $v_0$ making an angle $\theta$ with the horizontal. 
        If the angles of the vertices are $\alpha$ and $\beta$, figure out a relation 
        between $\tan \theta$, $\tan \alpha$ and $\tan \beta$.

        \begin{soln}
            Divide the range of the projectile into two lengths, $x$ and $x'$ such that $x + x' = R$. Now, clearly,
            \[
            \tan \alpha = \frac{y_0}{x} \qquad \tan \beta = \frac{y_0}{x'}
            \]
            
            Also, by the alternate equation of trajectory,
            \[
            y_0 = x \tan \theta \left(1 - \frac{x}{R} \right) = x \tan \theta \left(1 - \frac{x}{x + x'} \right)
            \]
            
            Which gives us,
            \[
            \tan \theta = y_0 \left( \frac{x + x'}{x x'} \right) = \frac{y_0}{x} + \frac{y_0}{x'}
            \]
            
            Finally substituting the values, we get,
            \[
            \tan \theta = \tan \alpha + \tan \beta
            \]
        \end{soln}

    \end{example}
    
    There are two important properties of projectile motion. One is more general, and that is that for 
    classical mechanics, motion is time-reversible. That is, if we do the transformation $t \to -t$, the motion 
    doesn't change. The idea is that the acceleration remains invariant, and the velocity reverses, so the trajectory 
    is the same, for forces of the form $F(x)$. We will say more about this later, when newton's laws are introduced, but 
    till then you can use it as a fact.

    The second is that the time to attain some vertical displacement can be found in terms of the maximum height alone,
    in fact the motion is independent of range. 

    Consider the vertical position of the particle to be $s$ at some time $t$. Then,
    \begin{equation*}
        s = u_y t - \frac{1}{2}gt^2
    \end{equation*}

    Also, note that if the maximum height is $h$,
    \begin{equation*}
        u_y = \pm \sqrt{2gh}
    \end{equation*}

    Where the positive value is during the ascent and the negative is during the descent. Then,
    \begin{equation*}
        s = \pm \sqrt{2gh}t - \frac{1}{2}gt^2
    \end{equation*}

    Solving for $t$ gives us,
    \begin{equation}
        t = \pm \sqrt{\frac{2h}{g}} \pm \sqrt{\frac{2(h - s)}{g}}
    \end{equation}

    If the initial velocity is positive and $s > 0$,
    \begin{equation}
        t = \sqrt{\frac{2h}{g}} \pm \sqrt{\frac{2(h - s)}{g}} 
    \end{equation}
    the two values correspond to the fact that the particle will have the same vertical displacement at two instances of time.

    If the initial velocity is positive but $s < 0$, i.e., the particle starts at some height but then transverses below it,
    \begin{equation}
        t = \sqrt{\frac{2h}{g}} + \sqrt{\frac{2(h - s)}{g}} 
    \end{equation}
    is the only reasonable solution.

    If the initial velocity is negative, then clearly the particle will attain a lower position. The maximum height in this case is simply the initial height of the particle.
    \begin{equation}
        t = -\sqrt{\frac{2h}{g}} + \sqrt{\frac{2(h - s)}{g}} 
    \end{equation}

    Thus qualitatively, projectile motion is independent of its horizontal range. 
    If some projectiles have the same maximum height, their time of flight \textit{will be the same}.

    \newpage

    \begin{marginfigure}
        \vspace{2em}
        \hspace*{-3.3em}
        \vspace{-2em}
        \scalebox{5}{\incfig{projectile-vec}}
        \caption{Projectile with Vectors}
        \label{fig: projectile-vec}
    \end{marginfigure}

    \begin{marginfigure}
        \hspace{-3.2em}
        \vspace{-2em}
        \scalebox{3.7}{\incfig{wedge-projectile-distance}}
        \caption{Projectile on tilted ground}
        \label{fig: projectile-titled-ground}
    \end{marginfigure}

    \begin{exc}
            \begin{exercise}[subtitle={Projectile with Vectors.}, points = 3]
            \\
            We can integrate the kinematic equations vectorially, to get

            \[
            \mathbf{v}(t) = \mathbf{v}_0 + \mathbf{g}t,
            \]
            
            and for the displacement,
            \[
            \mathbf{s}(t) = \mathbf{v}_0 t + \frac{1}{2} \mathbf{g} t^2.
            \]
            
            \begin{enumerate}
                \item[(a)] Show that these two equations are, in fact true. Justify our integration of a vector when 
                discussing velocity and acceleration. \emph{Hint: How can we integrate a vector? Try to use its components.}
            
                \item[(b)] Note that the vectors $\mathbf{v}_0 t$, $\mathbf{g} t^2 / 2$ are parallel to vector $\mathbf{v}_0$ and $\mathbf{g}$ respectively. 
                Use this to find the range and time of flight. \textit{Hint: You will find \cref{fig: projectile-vec} useful.}
            \end{enumerate}
        \end{exercise}

        \begin{exercise}[subtitle={Projectile Motion in tilted axes.}, points = 3, ID=projectile-tilted]
            \\
            Consider the case of a projectile launched along tilted ground (or a wedge), as in, fig.~3.7. 
            If we were to figure out $d$, we could use a number of ways. The equation of the projectile is,
            \[
            y = x \tan(\theta + \alpha) - \frac{g x^2}{2 v_0^2 \cos^2(\theta + \alpha)}
            \]

            and of the wedge is,
            \[
            y = x \tan \theta
            \]

            We could simply equate them to figure out the distance $d$, which is the range of the projectile. 
            However, let us look at alternate way of doing this, using tilted axis. This is the first time 
            we're using the \emph{transformation} of our reference frame. There are many more to come after this.

            \begin{enumerate}
                \item[(a)] Rotate the axes by the angle of inclination of the wedge, $\theta$, so that the $x$ axis 
                lies along the plane, and $y$ axis perpendicular to it. Now, find the kinematic equations along the two axes.

                \item[(b)] Solve these to find the time at which the projectile hits the wedge, and the distance it travels.
            \end{enumerate}
            \label{exc: projectile-tilted}
        \end{exercise}
    \end{exc}


    \section{Polar Co-ordinates}

    The polar co-ordinate axes are the two-dimensional subset of the 3-d cylindrical co-ordinates. Any point 
    in polar co-ordinates is depicted by a system of two unit vectors, $\rhat$ and 
    $\that$. However, each of these are dependent on the 
    position of the particle and may be written as $\rhat(\theta)$ 
    and $\that(\theta)$.

    \begin{marginfigure}
        \hspace{-5.7em}
        \vspace{-3em}
        \scalebox{4.5}{\incfig{polar-coords}}
        \caption{Polar-coordinates}
        \label{fig: polar}
    \end{marginfigure}

    Here, $\rhat$ points in the direction of increasing radius 
    (along the radial vector), and $\that$ points in the 
    direction of increasing $\theta$ (tangent to the radial vector). 
    These two unit vectors are \textit{orthogonal} at any point.

    In \cref{fig: polar}, if $\rhat$ makes an angle $\theta$ with the 
    horizontal, then,
    \begin{subequations}
        \label{eqn: polaruvec}
        \begin{alignat}{2}
            \rhat &= \cos\theta \ihat + \sin\theta \jhat \\
            \that &= -\sin\theta \ihat + \cos\theta \jhat         
        \end{alignat}
    \end{subequations}
    Which follows by adding projections of $\ihat$ and $\jhat$ along the radial and tangential directions.

    Now let us formulate kinematics in polar co-ordinates. The 
    position vector $\vec{r}$ can be written as
    \begin{equation}
    \vec{r} = r \rhat
    \end{equation}

    Velocity follows normally,
    \begin{equation*}
    \vec{v} = \dv{\vec{r}}{t} = \dot{r} \rhat + r 
    \dv{\rhat}{t} 
    \end{equation*}

    Or does it? What even is $\dv*{\rhat}{t}$? The answer 
    lies in the time-derivative of the unit vectors. Using the 
    definitions in \cref{eqn: polaruvec}, we get:
    \begin{subequations}
        \begin{alignat}{2}
            \dv{\rhat}{t} &= \dot{\theta} \that  \\
            \dv{\that}{t} &= -\dot{\theta} \rhat         
        \end{alignat}
    \end{subequations}

    Using these results, we get that
    \begin{align}
    \vec{v} &= \dot{r} \rhat + r \dot{\theta} \that \\
    \vec{a} &= (\ddot{r} - r \dot{\theta}^2) \rhat + 
    (r \ddot{\theta} + 2 \dot{r} \dot{\theta}) \that 
    \end{align}

    The most interesting result is the various terms of 
    acceleration in (3.32). First, let us consider the terms along 
    $\rhat$. The term $\ddot{r}$ is the \textit{radial 
    acceleration}. It is the change in the radial speed (and thus 
    the radius). The second term $-r\dot{\theta}^2$ is the 
    \textit{centripetal acceleration}. This term accounts for 
    change in direction of tangential velocity and is responsible 
    for the particle’s curved path.

    Now consider the terms along $\that$. The term 
    $r \ddot{\theta}$ is the \textit{tangential acceleration} due 
    to change in tangential speed:
    \[
    a_t = r \ddot{\theta} = \dot{v}
    \]

    The other term $2 \dot{r} \dot{\theta}$ is the 
    \textit{coriolis acceleration}, which we’ll discuss with 
    rotating frames.

    \begin{example}
        The movement of a particle in a circle about some point is called circular motion. If we move to the 
        origin, $r$ does not change with time. Thus, $\dot{r} = \ddot{r} = 0$. This gives us $v = \dot{\theta}r$. 
        $\dot{\theta}$ is often refered to as the \vocab{Angular Speed}, $\omega$. 
        
        We also find that
        $a_r = -m\omega^2r$ and $a_\theta = r\ddot{\theta}$. If $\ddot{\theta} = 0$, then the motion is called 
        uniform circular motion, and the acceleration is perpendicular to the velocity.   
    \end{example}

\section{Optimization Problems}
    These are a class of math problems with almost no physics involved. The following problem, for instance, is a purely
    geometrical one. These problems rely on tricks, and can be fun puzzles.
    \begin{question}
        A man who can run at a constant speed stands at a point with coordinates $(4, 5)$ and has to walk to the $x$-axis, then to the point $(3, 7)$. What path must the man walk along?
    \end{question}

    A large subset of these questions can be done purely by calculus, but that is not always the most elegant way to go 
    about it. We'll cover questions of both kinds, sometimes you do need to do the effort and the elegant solution
    may not be so elegant.
        
    \begin{marginfigure}
        \vspace{20em}
        \hspace*{-1.5em}
        \scalebox{2.5}{\incfig{fermat-principle}}
        \caption{One of the possible paths to reach from $A$ to $B$.} 
    \end{marginfigure}

    \begin{example}
        A man can run at a speed $u$ and swim at a speed $v$. In the following diagram, 
        the man starts at point $A$ and has to save his friend from drowning at point $B$. 
        Assuming he likes his friend, what should he do? 

        \begin{soln}
            The man should call a lifeguard. Now, let's figure out how the lifeguard should run from point $A$ to $B$.
            The time taken to reach the man is: 
            \[
                t = \frac{\sqrt{x^2 + h^2}}{u} + \frac{\sqrt{(d-x)^2 + h'^2}}{v}
            \]
            where $d$ is the horizontal distance between $A$, $B$, and $h$, $h'$ is vertical 
            distance of $A$ and $B$ to the shore, and hence independent of the path taken.

            The derivative of this with respect to the $x$ has to be $0$ for the best path of the lifeguard:
            \[
                \frac{x}{u\sqrt{x^2 + h^2}} = \frac{(d-x)}{v\sqrt{(d-x)^2 + h'^2}}
            \]
            But $\sin\alpha = x/\sqrt{x^2 + h^2}$ and $\sin\beta = (d-x)/\sqrt{(d-x)^2 + h'^2}$, so we have,
            \begin{equation}
            \boxed{\frac{u}{\sin\alpha} = \frac{v}{\sin\beta}}
            \end{equation}
        \end{soln}
    \end{example}


    This path follows the laws of refraction. In fact, this hints at a more general result for light, 
    that being that it follows the path of least time. This is called Fermat's principle.
     Of course, we don't care about what light does at this point, but it is a very cool 
     result. Note for one that we paid no attention to which reference frame we were talking about, and it 
     doesn't matter (unless its rotation).

    As you may notice, the principle doesn't really depend on what the media or distances are,
    in fact this is a general result.  

    A beautiful application of this principle is the Brachistochrone problem, which is given at the end of this section. 

    \begin{example}
        \label{ex: max-wedge-projectile}
        Consider a projectile thrown with a velocity $v_0$ making an angle $\alpha$ with horizontal, on wedge of inclination 
        $\theta$, which slopes downwards. Find out the angle $\alpha$ such that the projectile has maximum range.

        \begin{soln}
            The locus of the projectile is,
            \[
            y = x \tan \alpha - \frac{g x^2}{2 v_0^2 \cos^2 \alpha}
            \]
            and of the wedge is,
            \[
            y = -x \tan \theta
            \]
                    
            At the range, they will be equal so,      
            \[
            x \tan \alpha - \frac{g x^2}{2 v_0^2 \cos^2 \alpha} = 
            -x \tan \theta 
            \Longrightarrow \tan \alpha + \tan \theta = 
            \frac{g x}{2 v_0^2 \cos^2 \alpha}
            \]
            which gives us the desired range,      
            \[
            x = \frac{2 v_0^2 (\tan \alpha + \tan \theta) 
            \cos^2 \alpha}{g}
            \]
                    
            So the condition for maximum range is simply,
            \[
            \dv{x}{\alpha} = 0.
            \]
            This also kind of illustrates a nice idea, since we need to get 
            the derivative 0, we can safely discard any multiplicative or 
            additive constants right away which will either have a zero 
            differentiation, or can be divided by since the other side of 
            the equation is simply 0.
                    
            We get that,
            \begin{align*}
                &\dv{}{\alpha} \left\{ 
                    (\tan \alpha + \tan \theta) \cos^2 \alpha 
                \right\} = 0 \\
                &\dv{}{\alpha} \left\{ 
                    \frac{\sin 2\alpha}{2} + \tan \theta 
                    \cos^2 \alpha 
                \right\} = 0 \\
                &\cos 2\alpha - \tan \theta \sin 2\alpha = 0 \\
                &\tan 2\alpha = \cot \theta \\
                &\tan 2\alpha = \tan\left( \frac{\pi}{2} - \theta \right) \\
                \implies &\alpha = \frac{\pi}{4} - \frac{\theta}{2}
                \end{align*}
             \end{soln}
             
             An easier way to find $x$ is to use \Cref{exc: projectile-tilted}, and then we can continue in the same
             manner.
    \end{example}


    We'll see this idea in the problems, but when the accelerations or velocities are complicated, but it is often helpful 
    to consider the \emph{velocity space} or the position space, which is just the vector space of all velocities, instead 
    of our usual position space.

    \begin{example}
        A bug wishes to jump over a cylindrical log of radius $R$ lying on the ground, so that it just
        grazes the top of the log horizontally as it passes by. What is the minimum launch speed $v$
        required to do this?

        \begin{soln}
            First of all, since the the bug grazes the top of the log horziontally, $v_y$ at the top is just 
            $0$. Since $v^2 = v_0^2 - 2gh$, this implies that $v_{0,y} = \sqrt{2gh}$. Now we just need to find $v_x$.
            
            The idea here is of radius of curvature. If $v_x$ is too low, the
            trajectory of the bug will just pass through the log. If it's too high, our velocity is not minimal. 
            At the most optimal trajectory, the \emph{curvature} of the trajectory is the same as that of the log, so
            the bug is doing circular motion at the top of the log with radius $r$. Using polar-cordinates, and $\dot{r} = 0$, we find that 
            $v_x = gR$, since $\vec{g}$ points radially at top.

            Thus, $v_{\text{min}} = \sqrt{v_y^2 + v_x^2} = \sqrt{5gR}$.
        \end{soln}
    \end{example}


    \begin{exc}
        \begin{exercise}[subtitle={Projectile from a height}, points = 3]
            \smallskip
            ~\\
            Consider a projectile thrown from height $h$ with velocity $v$. What is its maxium horizontal range?
            There's two ways we will tackle this problem:

            \begin{enumerate}
                \item[(a)] Write down the equation of trajectory, but shifted so that $y \to y - h$. At $x = R$, the range,
                 we should have $y = 0$ (or don't change $y$, but note that projectile motion is time reversible, so $y=h$ at 
                 $x = R$). Then take the derivative of it with respect to $\theta$, while noting that for optimal trajectory 
                 $\dv*{R}{\theta} = 0$. Find $\theta$ and then $R$.
                 \item[(b)] Again write down the same shifted equation of trajectory. Now, write down every trignometric function
                 in terms of $\tan\theta$. You should get a quadratic. For such a trajectory to exist, $\tan\theta$ must have 
                 real roots. Use this to get $R$. 
            \end{enumerate}
        \end{exercise}

        \begin{exercise}[subtitle={River and Drift}, points = 2]
            \smallskip
            ~\\
            A swimmer can swim with velocity $u$ in still water. If the river has a velocity $v$, what is the minimum 
            distance through which she can cross the river? We'll do this in two ways as well.
            \begin{enumerate}
                \item[(a)] Let the velocity of the swimmer make an angle $\theta$ with the vertical. The distance perpendicular 
                to stream does not change with $\theta$, so we can ignore it. Calculate and minimise then, the distance 
                parallel to the flow of the river.
                \item[(b)] Draw the \emph{velocity} space of the resultant velocity of $u$ and $v$. That is, by varying $\theta$,
                consider how the resultant changes geometrically. Do this for both $v > u$ and $u \ge v$. 
            \end{enumerate}
        \end{exercise}

        \begin{exercise}[subtitle={Brachistochrone}, points = 4]
            \smallskip
            ~\\
            Suppose we drop a ball from a point $A$ to point $B$, vertically $h$ and horizontally $d$ apart along a 
            curve. Find the curve which minimises the time of descent. You might find $v^2 = v_0^2 + 2g\Delta h$ helpful.
        \end{exercise}

        \begin{exercise}[subtitle={Jumping over Roofs, Kalda}, points = 3]
            \smallskip
            ~\\
            Two fences of heights $h_1$ and $h_2$ are erected on a horizontal plain, so that the tops
            of the fences are separated by a distance $d$. Show that the minimum speed needed to throw a
            projectile over both fences is $\sqrt{g(h_1 + h_2 + d)}$.
            \label{exc: kaldaroof}
        \end{exercise}

        \begin{exercise}[subtitle={Perpendicular Velocities, Morin}, points = 3]
            \smallskip
            ~\\
            In the maximum-distance case of \Cref{ex: max-wedge-projectile}
            show that the initial and final velocities are perpendicular to each other. \emph{Hint: Use the fact 
            that projectile motion is time reversible.}
        \end{exercise}

        \begin{exercise}[subtitle={Parabolic Envelope, Knzhou}, points = 5]
            \smallskip
            ~\\
            The boundary of the set of all points a projectile with fixed velocity can reach by varying $\theta$ is 
            a parabola, with the focus at launch point. We can use this to geometrically solve many problems.
            \begin{enumerate}
                \item[(a)] Show that the trajectory that touches any point on the parabola must be tangent to it.
                \item[(b)] Show that the velocity at the point of tangency must be perpendicular to the initial one, if 
                we reach the point on the parabola with minimum velocity. Use this to solve the previous problem.
                \item[(c)]  Find the optimal angle in \Cref{ex: max-wedge-projectile} and solve \Cref{exc: kaldaroof}
                using the envelope and (a), (b). 
            \end{enumerate}
        \end{exercise}

        \begin{exercise}[subtitle={Projectiles with Vector, V2.}, points = 3]
            \smallskip
            ~\\
            When a projectile thrown along a wedge hits it, the sum of the scaled velocity and acceleration vector 
            must equal the vector along the slope with the magnitude equal to the distance the projectile travels 
            along the wedge. 
            
            Using this, the vector equations for $\vec{v}$, $\vec{r}$ and part (b) of the previous problem,
            show that $\vec{v}_0$ lies along the line joining focus and 
            projection of point at which it hits the wedge on the directrix, and find the angle using some angle chase.
            
            Do not use the reflexive property of parabola, in fact, this gives you a proof for it.
        \end{exercise}

        \begin{exercise}[subtitle={Tired Flappy Bird, OPhO}, points = 4]
            \smallskip
            ~\\
            A flappy bird can jump multiple times in the air. Each time it jumps
            mid-air, it can suddenly change its speed and direction. For every jump, the bird can decide when to jump
            and in which direction. Between jumps, the bird falls freely under gravity, which pulls it down at the
            acceleration $g$. 
            
            Say, our tired flappy bird starts off the cliff of height $H$ with the jumping velocity $V [1] = V_0$.
            Subsequent jumps in mid-air have decreasing velocities, i.e. the $n$-th jump has speed $V [n] = V_0/n$, $(n > 1)$.
            This majestic Vietnamese animal wants to travel as far as possible horizontally before it lands on the
            ground. Find the maximum horizontal distance the bird can travel.
        \end{exercise}
    \end{exc}


    \section{Frames and Symmetry}  
    
    Often in a mechanics problem, switching frames proves to be very beneficial. It is especially useful 
    if we have two moving objects, and we move to the frame of one, or move to a frame that offers high 
    symmetry. Suppose we move to a frame with position vector $\vec{R}$ with respect to our original frame. 
    Call the original frame $S$ and the current one $S'$. If the position vector of an object in $S$ os $\vev{r}$
    and in $S'$ is $\vec{r}'$, then,
    \begin{equation}
        \vec{r} = \vec{r}' + \vec{R}        
    \end{equation}
    The derivative of this gives 
    \begin{equation}
        \vec{v} = \vec{V} + \vec{v'}
    \end{equation}
    where $\dot{\vec{R}} = \vec{V}$ is the velocity of $S'$ with respect to $S$.
    
    We can go further, describing how accelerations are related, but that's not quite useful for kinematics, and 
    is a whole chapter in itself. So we'll leave that for now. Also note that $S'$ is not rotating.

    \begin{exc}
        \begin{exercise}[subtitle={Symmetrical Circles, Kalda}, points = 2]
            \smallskip
            ~\\
            Two intersecting circles of radius $r$ have centers a distance $a$ apart. If one circle moves towards the other with 
            speed $v$, what is the speed of one of the points of intersection?
        \end{exercise}
        \begin{exercise}[subtitle={Rotating Circles, Kalda}, points = 3]
            \smallskip
            ~\\
            Two circles of radius $r$ intersect at the point $O$. One of the circles rotates
            about the point $O$ with constant angular speed $\omega$. The other point of intersection $O'$ is originally a
            distance $d$ from $O$. Find the speed of $O'$ as a function of time.

            A point rotating with angular speed $\omega$ about $O$ just means in the frame in which $O$ is the origin, 
            and in polar co-ordinates, $\dot{\theta} = \omega$. 
        \end{exercise}
    \end{exc}

    \section{Invariants}

    Invariants are universal across physics. These give us \emph{conservation laws}, many of which we'll encounter in the 
    later chapters.

    An \vocab{Invariant} is a quantity that remains constant throughout time. Identifying an invariant in a problem is very
    helpful, and can help us solve some seemingly impossible problems. 

        \begin{example}
            A cat moves with speed $v$ always directed towards a rat moving rectilinearly with speed $v$. They are initially at a distance $\ell$ from each other, 
            and their initial velocities are perpendicular. After a long time they move along the same line, seperated
            by a distance $d$. Find $d$ 
            
            \begin{soln}
                    Solving this problem using the techniques we have already discussed is incredibly hard. 
                    Whenever you do no know what to do, writing all the equations governing the system isn't a bad idea.
                    
                    As I mentioned in the previous section, it might be good to move into the frame of one 
                    of them. Let's move into the frame of the fox. Let's orient our axis so that the seperation initially 
                    was along the $y$ axis, and thus the cat moves with $v\ihat$. 

                    Consider the system after time $t$, as in \Cref{fig: fox-chase}. We don't really 
                    say anything nice in cartesian co-ordinates alone. But note that the velocity of the rat (aside from the 
                    horizontal one) because of switching the frames must point radially. So let's try out 
                    polar-coordinates. 
                    
                    The velocity along the $x$ axis 
                    is $v - v\cos\theta$. The velocity along the radial direction is $v\cos\theta - v$. Thus,
                    \[
                        \dv{x}{t} + \dv{r}{t} = v - v\cos\theta + v\cos\theta - v = 0 \implies \dv{(x + r)}{t} = 0.
                    \]
                    So $x + r$ is a constant, independent of time. At $t = 0$, $r$ is just $\ell$ and $x = 0$. After a long 
                    time, $x = r$. So, $d = r = \ell/2$. 
            \end{soln}
        \end{example}

    \begin{marginfigure}
        \vspace{-18em}
        \scalebox{2}{\incfig{fox-chase}}
        \caption{Velocities and distances in the frame of cat}
        \label{fig: fox-chase}
    \end{marginfigure}
        
It seems almost impossible to solve the previous problem without invariants, and in fact trying to solve without invariants 
is very difficult. You can look at \href{https://arxiv.org/pdf/1108.2006}{this paper} for an extensive discussion on 
the problem. We'll only do one problem for this here, since most would require some knowledge of dynamics. But the 
idea is very important, and I urge you to keep it in mind. 
\begin{exc}
    \begin{exercise}[subtitle={Chasing Problem}, points = 2]
        \smallskip
        ~\\
        Suppose the cat in the previous example has velocity $u > v$. How long does it take to catch the rabbit?
    \end{exercise}
    \end{exc}