\setlength{\chnumsep}{11.6em}
\chapter{Laws of Motion}

\begin{overview}
    Newton's laws lay out the foundation of mechanics, and present us with the laws of physics 
    that govern the everyday world. We'll later see that they aren't exact, but in the regime of common 
    every day tasks, they present us with the tools to analyse any mechanical phenomenon. We'll also go 
    through various applications of these laws along with the general discussion.
\end{overview}

\section{Introduction}

Classical mechanics is a deterministic theory, having known the state of the physical system at one instant,
it wishes to determine the evolution of system as time goes on. You may have heard of indeterminance in quantum 
mechanics through pop science, but remember that it doesn't invalidate classical mechanics. 

In general 
even a less correct theory is useful if we use it in the right regime. As long as we don't try to work 
in classical mechanics while our cars are moving at the speed of light or our balls are comparable to 
the size of atoms, we'll be fine.

Classical mechanics is almost entirely governed by Newton's laws. Using merely a few axioms, 
we can understand alot of the daily phenomenons that we see. Let's start with the first law. 

\section{The First Law}

\begin{enumerate}
    \item[\textbf{N1}] Left alone, a particle moves with constant velocity.
\end{enumerate}

There are various different ways to write it of course, in terms of zero net force and whatever, but to 
get the main physical insight from this we have to look a little deeper. First of all, what does it mean for 
a particle to be left alone? 

Well let's first take about things that are not left alone, for instance the planets. We know that each of 
them interact physically with the sun and rotate around it. And there are other examples, a ball being 
juggled, a cart being pushed, a nail being attracted by a magnet, etc. 

The common theme here is that each 
of the objects experiences a \emph{physical interaction} of some manner. Gravity, magnetic forces, 
the force we exert, all are results of a physical interaction between two bodies. 

And thus we expect that a body that if left alone does not experience any sort of physical interaction: it 
is \emph{isolated} from all other objects. 

This law doesn't really hold in all reference frames, but defines an inertial frame, which 
provide the setting in which newton's laws hold. We'll later see that a little modification of these laws 
will help us use them in any sort of frame, but we'll restrict ourself to inertial frames for now.

\begin{definition}
    An \vocab{Inertial Frame} is an observational frame of reference in which an isolated body moves 
    with constant velocity.
\end{definition}

The first laws then boils down to an axiom of existence, and that is the physical content of the law:

\begin{enumerate}
    \item[\textbf{N1}] Inertial frames exists.
\end{enumerate}

Thus the first law is part definition and part experimental fact. From experiments, we know that our 
axiom is in fact valid. But there is one issue here, can we always isolate a body? For instance suppose 
that the interaction between it and some object increased as distance increases: then we could never 
isolate it, no matter how far we went (going too close would give us all sort of other forces, like gravity). 

Fortunately, from what we know experimentally, this is not the case. All interactions eventually become
negligible as we increase the distance, so we can get an isolated body. Even without going to very large distances, 
we can manually cancel out the other forces to get such a body (for instance by using an air track).

In any case, the idea of an inertial frame is extremely important for classical mechanics, and we'll see 
it pop up quite a few times.

\section{The Second Law}

\begin{enumerate}
    \item[\textbf{N2}] The force on a body is proportial to the time rate of change of its momentum.
\end{enumerate}

Before talking about what forces are, let's define momentum:

\begin{definition}
    The \vocab{Linear Momentum} of a particle is:
    \begin{equation}
        \vec{p} \equiv m\vec{v}
    \end{equation}
\end{definition}

Thus it follows from the second law that the force, $\vec{F}$ is:

\begin{equation}
    \vec{F} = \dot{\vec{p}} = m\dot{\vec{v}} + \dot{m}\vec{v}.
\end{equation}

We choose our units so that the constant of proportionality is $1$.

For constant mass $\dot{m} = 0$ so that,

\begin{equation}
    \vec{F} = m\vec{a}.
\end{equation}

which is the form of the second law you are probably already acquainted with. There are a couple of problems 
here: what is $m$, and what do we really mean by a force?

First, let's talk about forces.

\subsection{Force}   

The section can be skipped without a loss of continuity.

From the second law it almost seems tempting to include that whatever 
creates an acceleration is a force. There's some merit to that idea but 
then the second law just becomes a definition—and how will something 
that we have purposefully defined as $\vec{p}$ tell us something about 
the physical laws that we don't already know?

First of all if we go be the intuitive definition of force, which you 
may have heard of as a “push or pull”, it is apparent that not all 
forces cause acceleration—you can push a wall but it probably won't 
move (if it does, a strongman might be one of the career paths you want 
to look into). But really then how do we go about “defining” what a 
force is?

The point of interest here is that force is not a mere definition, it 
\textit{exists independently of acceleration}. Whenever two bodies with 
some masses interact with each other gravitationally the force there is 
the gravitational force and characterises the particular interaction. 
The forces are specified by the model in which we are working.

Applying the same forces on two bodies should have the same effect as 
described in the second law. If $dm_i/dt = 0$, then:

\[
m_i a_i = m_j a_j.
\]

This \textit{can} be verified, that the ratio in which accelerations of 
the two bodies, that are produced by the same force is constant because 
$m_i$, $m_j$ do not change.

There is also the question on how we find the forces which act in our 
model—say what is the force between two stationary charged particles? 
The answer to that is well, using Newton's second law. Acceleration is 
measurable, and we know $m$ for a specific particle (we will talk about 
this in a second) so by varying different parameters, the distance 
between particles, the masses of each, the charge etc. we can measure 
their acceleration, and thus we can find how force scales with those 
factors.

This seems to almost invalidate our previous discussion—we are now 
measuring forces through acceleration when we just asserted that they're 
independent of it. That's not quite wrong, however, it is only because 
we know forces are independent of the particle or acceleration that we 
can now say the same sort of force acts even when the parameters are 
changed and how that force scales with the quantities. After we have 
found the force, we can again say because it is independent of the 
acceleration, it acts on any body under the same model and use the 
results we have found.

If you haven't been able to follow the discussion completely, don't 
worry, this doesn't affect your actual usage of the second law. We will 
also make do with forces later, turning our attention to energy instead 
but that is a discussion for another time.

Another way to look at it is while the second law alone does not define 
a force, the three laws can be understood as an axiom system for 
defining what a force is. We don't really need an explicit definition in 
that case, many axiom systems are similar.

\sidenote{If you wish to read more about this, consider \url{https://archive.org/details/conceptsofforces00jamm/page/260/mode/2up} and \url{https://www.math.cmu.edu/~wn0g/Force.pdf}.}

In any case, using the three laws as axioms, all our preceding discussion tells us is how to experimentally verify
the second law, and how physical models independetly carry the information about what the force we are looking for is 
(rather than what forces in general are).

About the whole whatever causes acceleration is a force thing by the way, that is also not
exactly true. Suppose that a block is accelerating to the right with some acceleration. Is the
force on it equal to ma and towards the right? Not quite, the actual force maybe in whatever
direction, insofar that it has a component in the direction of the acceleration, and may have
some other value ``$ma$'' is not the force. 

The force is once again, whatever arises out of actual
physical interaction, and what we exists in our model.
It might be tempting to ask in case of forces where such an interaction is not visible, say
gravity, does this physical interaction really exist? In fact, it does the interaction is mediated
through the use of fields.

Sometimes we may also be concerned with forces arising due to macroscopic effects of small
interactions. We do not thus, consider them by first principle and repeatedly apply the laws of
our fundamental forces, but instead make empirical laws to avoid complexity, like the normal,
tension, spring, etc. forces.

\subsection{Mass}

Assume that there are two bodies, of masses $m_1$ and $m_2$. Suppose 
they are under the same force. In this example, in particular we can 
consider a very elementary idea of the deformation in springs.

By experimental results, we know that the deformation in spring is 
proportional to the force exerted by it. Consider a setup in which we 
connect a carriage of negligible mass which is connected to a spring; 
by a rod to a cart which we can accelerate at any acceleration freely.

\begin{marginfigure}
    \begin{tikzpicture}[scale=1.2]

  % Wall
  \fill[black] (0,0) rectangle (0.2,3);

  % Spring
  \draw[decorate, decoration={aspect=0.3, segment length=3mm, amplitude=3mm, coil}] 
    (0.2,2) -- (2.5,2);

  % Carriage (mass m1)
  \fill[black] (2.5,1.6) rectangle (3.1,2.4);
  \node[right] at (3.1,2.0) {\emph{Carriage}, $m_1$};

  
  % Cart
  \fill[black] (0.2,0.55) rectangle (1,0.75);
  \fill[black] (1,0.4) rectangle (1.6,0.9);
  \node[right] at (1.6,0.65) {\emph{Cart}};

\end{tikzpicture}
\caption{A cart connected to a carriage on an air track. As we
accelerate the cart, we also accelerate the carriage and compress
the spring until we achieve our desired compression length. The spring force mediates the force
which we use to pull the cart to the carriage.}
\end{marginfigure}

We place the setup on an air track, which is a one-dimensional track 
that has holes blowing out air to ensure that the effect of forces such 
as friction is negligible on the body.

Suppose we place a mass $m_1$ in the carriage, and accelerate the cart 
to acceleration $a_1$ so that the carriage too accelerates with $a_1$. 
The only force experienced by the mass is the spring force, which causes 
it to accelerate with acceleration $a_1$. We wait till the spring stops 
compressing, and note down the final length, call it $\ell$.

Now, we can repeat this experiment for a mass $m_2$ and get some 
acceleration $a_2$ by adjusting the acceleration of the cart such that 
the final length of the spring when it stops compressing becomes $\ell$.

Since we claimed that the force exerted by a spring is proportional to 
its compression, the force experienced by a mass $m_1$ accelerating at 
$a_1$ is equal to the force experienced by $m_2$ accelerating at $a_2$. 
So, $m_1 a_1 = m_2 a_2$. And taking the $m_1$ as the unit mass, we define,

\[
m_2 \equiv m_1 \frac{a_1}{a_2}.
\]

We define the mass of the first body relative to the other. The 
accelerations can be absolutely measured using an accelerometer. In such 
a manner we define the acceleration of any $i^{th}$ body relative to the 
first one as,

\[
m_i \equiv m_1 \frac{a_1}{a_i}.
\]

We define some standard mass as 1 unit and then define the other masses 
relative to this. Such a definition is an \emph{operational definition}.

\section{The Third Law}

\begin{enumerate}
    \item[\textbf{N3}] The force on a body \(B\) due to another body \(A\) is equal and opposite to the force experienced
    by \(A\) due to \(B\). 
\end{enumerate}

The third law is also known as the weak law of action and reaction— in direct contrast with
the strong law which requires the equal and opposite forces to lie on the same line of action.
Most forces such as gravity and the normal force in fact obey the strong law of action and
reaction. We will see later, when we deal with rotations, that all forces not obeying the strong law causes some headache.