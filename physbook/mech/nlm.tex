\setlength{\chnumsep}{11.6em}
\chapter{Laws of Motion}

\begin{overview}
    Newton's laws lay out the foundation of mechanics, and present us with the laws of physics 
    that govern the everyday world. We'll later see that they aren't exact, but in the regime of common 
    every day tasks, they present us with the tools to analyse any mechanical phenomenon. We'll also go 
    through various applications of these laws along with the general discussion.
\end{overview}

\section{Newton's Laws}

Classical mechanics is a deterministic theory, having known the state of the physical system at one instant,
it wishes to determine the evolution of system as time goes on. You may have heard of indeterminance in quantum 
mechanics through pop science, but remember that it doesn't invalidate classical mechanics. 

In general 
even a less correct theory is useful if we use it in the right regime. As long as we don't try to work 
in classical mechanics while our cars are moving at the speed of light or our balls are comparable to 
the size of atoms, we'll be fine.

Classical mechanics is almost entirely governed by Newton's laws. Using merely a few axioms, 
we can understand alot of the daily phenomenons that we see. Let's start with the first law. 

\subsection{The First Law}

\begin{enumerate}
    \item[\textbf{N1}] Left alone, a particle moves with constant velocity.
\end{enumerate}

There are various different ways to write it of course, in terms of zero net force and whatever, but to 
get the main physical insight from this we have to look a little deeper. First of all, what does it mean for 
a particle to be left alone? 

Well let's first take about things that are not left alone, for instance the planets. We know that each of 
them interact physically with the sun and rotate around it. And there are other examples, a ball being 
juggled, a cart being pushed, a nail being attracted by a magnet, etc. 

The common theme here is that each 
of the objects experiences a \emph{physical interaction} of some manner. Gravity, magnetic forces, 
the force we exert, all are results of a physical interaction between two bodies. 

And thus we expect that a body that if left alone does not experience any sort of physical interaction: it 
is \emph{isolated} from all other objects. 

This law doesn't really hold in all reference frames, but defines an inertial frame, which 
provide the setting in which newton's laws hold. We'll later see that a little modification of these laws 
will help us use them in any sort of frame, but we'll restrict ourself to inertial frames for now.

\begin{definition}
    An \vocab{Inertial Frame} is an observational frame of reference in which an isolated body moves 
    with constant velocity.
\end{definition}

The first laws then boils down to an axiom of existence, and that is the physical content of the law:

\begin{enumerate}
    \item[\textbf{N1}] Inertial frames exists.
\end{enumerate}

Thus the first law is part definition and part experimental fact. From experiments, we know that our 
axiom is in fact valid. But there is one issue here, can we always isolate a body? For instance suppose 
that the interaction between it and some object increased as distance increases: then we could never 
isolate it, no matter how far we went (going too close would give us all sort of other forces, like gravity). 

Fortunately, from what we know experimentally, this is not the case. All interactions eventually become
negligible as we increase the distance, so we can get an isolated body. Even without going to very large distances, 
we can manually cancel out the other forces to get such a body (for instance by using an air track).

In any case, the idea of an inertial frame is extremely important for classical mechanics, and we'll see 
it pop up quite a few times.

\subsection{The Second Law}

\begin{enumerate}
    \item[\textbf{N2}] The force on a body is proportial to the time rate of change of its momentum.
\end{enumerate}

Before talking about what forces are, let's define momentum:

\begin{definition}
    The \vocab{Linear Momentum} of a particle is:
    \begin{equation}
        \vec{p} \equiv m\vec{v}
    \end{equation}
\end{definition}

Thus it follows from the second law that the force, $\vec{F}$ is:

\begin{equation}
    \vec{F} = \dot{\vec{p}} = m\dot{\vec{v}} + \dot{m}\vec{v}.
\end{equation}

We choose our units so that the constant of proportionality is $1$.

For constant mass $\dot{m} = 0$ so that,

\begin{equation}
    \vec{F} = m\vec{a}.
\end{equation}

which is the form of the second law you are probably already acquainted with. There are a couple of problems 
here: what is $m$, and what do we really mean by a force?

