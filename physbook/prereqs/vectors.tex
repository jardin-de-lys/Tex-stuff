\setlength{\chnumsep}{14em}
\chapter{Vectors}

\begin{overview}
In this chapter we'll give an introduction to vectors, and some results in
geometry that you may find useful. We'll first introduce them geometrically, then
give a rigorous algebraic treatment of them. Vectors are very useful quantities,
that will aid us all the way to Quantum Mechanics, so being familiar with them will aid you 
a lot. 
\end{overview}


\section{Geometrical Vectors}

We will somewhat loosely follow the vectors chapter from \url{https://www.damtp.cam.ac.uk/user/sjc1/teaching/VandM/notes.pdf}.


We first describe vectors in a geometrical fashion. This isn't in anyway rigorous, and we won't 
pretend it is. We'll give some proofs, but you may need to take a leap of faith in some places.

\begin{marginfigure}
\incfig{freevector}
\caption{A directed line segment, the vector \(\vec{A}\).}
\label{fig: freevector}
\vspace{4em}
\end{marginfigure}

\begin{definition}
  A \vocab{Vector} is a quantity described by a magnitude and a direction in space. They're
  represented as line segments, as in \cref{fig: freevector}.  
\end{definition}
You can think of a vector as a \emph{directed line segment}.
Vectors are usually denoted using boldface, \(\vec{A}\). 

Because they're directed line segments, only their length matters and
we’re not concerned with the location of their end points. This means that a
vector can be freely translated, without altering it. An equivalent way to say 
this is that two vectors are equal iff their magnitude is equal, and they have the
same direction (i.e. they're parallel).

\begin{marginfigure}
  \scalebox{1.8}{\incfig{translatedvector}}
  \caption{Two identical vectors.}
  \end{marginfigure}

Also, we define a vector field, which is a special sort of function, that'll be relevant 
for future discussions.

\begin{definition}
  A \vocab{Vector Field} is a vector \(\vec{B}\) which is a function of position, \(\vec{B}(\vec{r}) = \vec{B}(x, y, z)\). 
\end{definition}

Where the position of a point \(P\) refers to the vector from the origin \(\mathcal{O}\) to the point, \(P\).

The magnitude of the vector is its length, which we call it's \tpvocab{Vector}{norm}, and write
as \(\norm{\vec{A}}\) or simply \(A\). You may also see it being written as \(\abs{\vec{A}}\).

A unit vector, \(\uvec{\gamma}\) (“gamma hat”) is a vector whose magnitude is unity. We use the unit vector to
often denote the direction of a vector by multiplying the unit vector with its magnitude.
For instance, the unit vector that point is the direction of \(\vec{A}\), \(\uvec{A}\) can be calculated as,
\[
\uvec{A} = \frac{\vec{A}}{A}.
\]

Further, vectors have certain operations defined on them. The base operations
are \tpvocab{Vector}{Scalar Multiplication} and \tpvocab{Vector}{Vector Addition}.

\subsection{Scalar Multiplication}  

Scalar Multiplication refers to multiplying a vector by a \vocab{scalar}. A scalar
for our purposes refers to a real number. Consider a vector \(\vec{v}\).
Multiplying it by a scalar, \(\alpha \in \RR\) produces a vector 
\(\alpha\vec{v}\) parallel to the original vector and changes the magnitude of the vector so that
\(\norm{\alpha\vec{v}}\) is \(\abs{\alpha}\) times greater than \(\norm{\vec{v}}\), thus
\(\norm{\alpha \vec{v}} = \abs{\alpha}\norm{\vec{v}}\). \(\abs{\alpha}\) can 
be smaller or greater than \(1\) which accordingly increases/decreases the 
length of \(\vec{v}\).

\begin{marginfigure}
  \centering
  \scalebox{1.5}{\incfig{scalar multiplication}}
  \caption{Scalar multiplication of a vector \(\vec{A}\) by \(c > 1\) and \(-1\).}
\end{marginfigure}

If \(\alpha > 0\), the vector produced is in the same direction as the original vector. If
\(\alpha < 0\), the direction of the vector is reversed. 

\subsection{Vector Addition}

Adding two vectors, \(\vec{A}\) and \(\vec{B}\) produces another 
vector \(\vec{A + B}\). Geometrically, vector addition is done by placing
the tail of \(\vec{B}\) on the head of \(\vec{A}\), and then the vector joining
the tail of \(\vec{A}\) and head of \(\vec{B}\) is the vector \(\vec{A + B}\) as 
in \cref{subfig: vectoradd1}.

It can also be interpreted as the diagonal of the parallelogram made by placing
the tails of the two vectors together, and producing two sides parallel to them as 
show in \cref{subfig: vectoradd2}. 

From this we may interpret that vector addition is commutative, that is, 
\(\vec{A + B} = \vec{B + A}\). We can also deduce that its associative,
\(\vec{A} + (\vec{B} + \vec{C}) = (\vec{A} + \vec{B}) + \vec{C}\). I'll 
recommend convincing yourself of this using geometrical constructions.

\begin{marginfigure}
  \centering
  \begin{subfigure}[t]{\marginparwidth}
    \scalebox{1.5}{\incfig{vectoradd1}}
  \caption{Adding vectors, parallelogram interpretation.}
  \label{subfig: vectoradd1}
  \end{subfigure}
  \begin{subfigure}[t]{\marginparwidth}
    \scalebox{1.2}{\incfig{vectoradd2}}
  \caption{Adding vectors, triangle interpretation.}
  \label{subfig: vectoradd2}
  \end{subfigure}
  \caption{Addition of two vectors, \(\vec{A}\) and \(\vec{B}\) produces another
  vector, \(\vec{A + B}\).}
\end{marginfigure}

Subtraction of vectors, \(\vec{A} - \vec{B}\) is equivalent to multiplying \(\vec{B}\) 
by \(-1\) and then adding it with \(\vec{A}\).

If \(\norm{\vec{v}} = 0\), we write \(\vec{v} = \vec{0}\). 

\subsection{Scalar Product}