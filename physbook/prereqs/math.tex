\setlength{\chnumsep}{8em}
\chapter{Mathematically Useful Things}

\section{Linear, Second Order, Homogenous, Time Translation Invariant ODEs}

\textit{Second order} means highest derivative in equation is ${d^2x}/{dt^2}$. An ODE is \textit{linear} if it doesn’t have any products of $x$, $\dot{x}$ and $\ddot{x}$ other than with a constant.
A ODE is \textit{homogeneous} if it only contains terms that are proportional to exactly one power of $x$ or its derivatives.
\textit{Time translational invariant} just means coefficients are independent of time.


These are ODEs of the form 
$$m \ddot{x} + b \dot{x} + kx = 0.$$

The general method to solve them is basically the same for all such equations. 
First, we elevate $x$ to a complex variable, $z$. 
The motivation for this is basically thus. If $z(t)$ satisfies the complex differential equation,
$$
\{m\Re(\ddot{z}) + b\Re(\dot{z}) + k\Re(z)\} + i\{m\Im(\ddot{z}) + b\Im(\dot{z}) + k\Im(z)\} = 0
$$
where $\Re(x)$ is the real part of $x$ and $\Im(x)$ is the imaginary part. Thus, we can write $z = \Re(z) + i\Im(z)$. The above equation follows. 

Now the idea is basically that if $a + bi = 0$, $a = b = 0$. 
This follows from equality of complex numbers (which is trivial to show). Thus, if we find $z(t)$, we can find $x = \Re(z)$ which satisfies our DE and is real. 
(Also $\Im(z)$, of course.) 

Anyway, the other thing is just, from the equation we want to work with exponentials, but that can be messy if for instance we have $$m \ddot{x} + kx = 0$$ with $k > 0$. A real exponential function $e^{\zeta t}$ gives you $\zeta^2 + k = 0$, which doesn’t have a real solution. So working in complex numbers is neater.

For our equation, we can just plug in an ansatz (which is a fancy way to say guess) $z(t) = Ae^{i\minusspace\omega\minusspace t}$, to get,

$$
-\omega^2 m + bi\omega + k = 0 \iff \omega = \frac{{bi \pm \sqrt{ 4km - b^2}}}{2m}
$$
The most general solution is an arbritrary superposition of $Ae^{i\omega t}$ for our two values of $\omega$. Thus,

$$
z(t) = e^{-bt/2m}\{Ae^{i\Omega t} + Be^{-i\Omega t}\}
$$
where $\Omega = \sqrt{4km-b^2 }/2m$, and $A, B \in \mathbb{C}$.

\subsection{1 $\Omega^2 > 0$}

Since $\Omega$ is real, we can take the real part to get,
$$
x(t) = e^{-bt / 2m}\{C\sin \Omega t + D\cos \Omega t\}
$$
where $C = \Im(B) - \Im(A)$ and $D = \Re(A) + \Re(B)$.

The amplitude roughly decreases like $e^{-bt /2m}$ (this is not exact because the factors affects the whole motion, not when we have our extremes, regardless the net change is described a constant times this).

This is called under-dampening in a case when $m$ is the mass of the spring, $k$ is its constant, and $bv$ is a dampening frictional force.


If $\Omega$ is small, then the oscillatory behaviour is not really visible, since the whole motion goes to $0$ in about ${} t = 2m/b < 2\pi/\Omega {}$ which is the time it takes to complete one oscillation. Of course, it is apparent that $\Omega > b/2m$ using $\Omega^2 > 0 \iff 4km > b^2$.

\subsection{2  $\Omega^2 < 0$}

In this case we can write $\Omega = i\tilde{\omega}$, with $\tilde{\omega} = \sqrt{ b^2 - 4km}/2m\in \mathbb{R}$. Thus,
$$
x(t) = e^{-bt /2m}\{Ce^{\tilde{\omega}t} + De^{-\tilde{\omega}t}\}
$$
with $C = \Re(B)$, D = $\Re(A)$. Also since $\Omega ^2 < 0 \iff 4km < b^2$, this implies $0 >-b/2m + \tilde{\omega} > -b/2m - \tilde{\omega}$, so the motion is of the form of a decaying exponential one. This is called overdampening in the same case as the one for underdampening.


\subsection{3  $\Omega^2 = 0$}

This is called critical dampening. Now first of all this poses a problem. If $\Omega = 0$, our solutions are not very nice (well kinda). In fact, in that case we only have one $\omega  = -bi/2m$. 

But this doesn’t actually work. The theory of differential equations tell us that our solution space is spanned by two basis vectors, $e^{i\minusspace\omega_{1}\minusspace t}$ and $e^{i\minusspace\omega_{2}\minusspace t}$ which are linearly independent. In this case our basis vectors are both the same, and not linearly independent! 

Physically this can be understood as having two basis vectors gives us a solution that is their linear combination. If they’re not linearly independent, our solution is of form $Ae^{i\minusspace\omega\minusspace t}$ which gives you only one free constant. But this means that for different velocities, the system behaves the same, which is unphysical (if we know $x(0)$, this solution determines $v(0)$, we aren’t free to choose).

A good way to look at it like this, we will write out our solution in a little different form,
$$
x(t) = Ae^{i\minusspace\omega_{1}\minusspace t} + Be^{i\minusspace\omega_{2}\minusspace t} = e^{i\minusspace\omega_{1}\minusspace t}\{A + Be^{i\minusspace(\omega_{2} - \omega_{1})t}\} = e^{i\minusspace\omega_{1}\minusspace t}\{A + B\{\cos (\Delta \omega t) + i\sin(\Delta \omega t)\}\}
$$
where $\Delta \omega = \omega_{2} - \omega_{1}$. For $\Delta\omega t \ll 1$, we can use the taylor expansion to get $\cos\Delta \omega t \approx {1}$ and $\sin\Delta \omega t \approx \Delta \omega t$. Write this out as
$$x(t) \approx e^{i\minusspace\omega\minusspace t}\{A + B + Bi\Delta \omega t\} = e^{i\minusspace\omega\minusspace t}\{C + Dt\}.$$

As $\Delta \omega \to 0$, that is we arrive at the same solutions, this behaviour persists for longer and longer, till we get the solution of the form,
$$
x(t) = e^{-b/2m} \{A + Bt\}.
$$

We can hand-wave the smallness of $\Delta \omega$ in the $B \Delta \omega t$ for now. This is just to give some form of intuition, and is not exactly what’s going on, but this should allow for a fair idea.

This is called critical dampening in the case of springs.

\section{$n$th order, homogeneous, time translational invariant, linear ODEs}

These are ODEs of the form,

$$
\left\{a_{n}\dv[n]{}{t} + \cdots + a_{1}\dv{}{t} + a_{0}\right\}x = 0
$$
The solutions are of the form,

$$
x(t) = A_{1}e^{i\minusspace\omega_{1}\minusspace t} + \cdots + A_{n}e^{i\minusspace\omega_{n}\minusspace t}
$$
If $\omega_{i} = \cdots = \omega_{j}$ are repeated, we substitute them with $e^{i\minusspace\omega_{i}\minusspace t}, te^{i\minusspace\omega_{i}\minusspace t}, \dots, t^{j-i -1}e^{i\minusspace\omega_{i}\minusspace t}$.

\section{Non Time Translational Invariant, Second order, Linear ODEs}

We’ll look at some very special kind of ODEs here,

$$
m \ddot{x} + b \dot{x} + kx = f(t).
$$