\setlength{\chnumsep}{10em}
\chapter{Electrostatics}

\begin{overview}
    Electrostatics is the study of fixed charges. In this chapter we'll cover one of maxwell's equation
    and just a whole lot about how electric forces interact with each other, even though 
    we won't quite talk about the motion of a charge particle.
\end{overview}

\section{Coloumb's Law}


\sidenote{We write $\vec{F}_{12}$ for the force on particle $2$ due to interaction between particle $1$ and $2$ and $\vec{\mathscr{r}} \equiv \vec{r}_2 - \vec{r}_1$, the vector 
from $1$ to $2$.}


The fundamental law governing interactions between two charged particles is \vocab{Coloumb's Law}.
The force on a particle of charge $q_1$ due to a particle of charge $q_2$, at positions $\vec{r}_1$ and $\vec{r}_2$ is:

\begin{equation}
    \boxed{\vec{F} = \frac{1}{4\pi\epsilon_0} \frac{q_1q_2}{\mathscr{r}^2} \uvec{\mathscr{r}}.}
\end{equation}

The coloumb forces similar to other forces we have seen now, follows the law of superposition. The force 
is parallel to the line joining both the charges, and from a simple symmetry argument we can see that 
it can be parallel to no other direction, atleast for stationary charges, since the only unique direction 
in this case is the line joining both the charges. Space is isotropic, so if it points in any other direction, we can 
simply rotate our whole setup with out altering anything to see that it must now point in a different direction.

If for instance the charges were moving, the above argument would not hold--- we could have some force in 
the direction perpendicular to acceleration and $\vec{\mathscr{r}}$ for instance. This would also be 
true if the charges had any structure, and so another intrinsic direction. But then we would need more quantities 
to talk about the charge than just the scalar $q$. This is true of the elementary particles (they have a 
property called the \emph{spin}), including 
electrons and leads to magnetic phenomena, and so does the movement of charges. In both cases the magnetic 
force does not in general point along the direction joining the two charges.

The constant for the coloumbic force, $1/4\pi\epsilon_0$ is introduced to historical reasons, where:
\[
\frac{1}{4\pi\epsilon_0} = 8.988 \cdot 10^9 \, \unit{\N\metre\squared\per\C\squared} \implies 
\epsilon_0 = 8.854 \cdot 10^{-12} \, \unit{\C\squared\per\N\per\metre\squared}
\]

Consider the forces due to charges $q_1$, $q_2$, $\ldots$, $q_n$ on a test charge $Q$:
\[
\vec{F} = \sum_{i = 1}^{N} \frac{1}{4\pi\epsilon_0}  \frac{q_i Q}{{\mathscr{r}_i}^2} \uvec{\mathscr{r}}_i
\]

To simplify matters, because we wish to find the force due to the system on any such charge, it seems quite 
fair to definine a quantity independent of the test charge. Thus we define the electric field, 
which is the property of the \emph{system} of charges.

\begin{definition}
    The \vocab{Electric Field} due to a charge $q$ at a point having position vector $\vec{\mathscr{r}}$ with
    respect to $q$ is:
    \begin{equation}
        \vec{E} \equiv \frac{1}{4\pi\epsilon_0} \frac{q}{\mathscr{r}^2}\uvec{\mathscr{r}}
    \end{equation}
\end{definition}

So that the force on a charge $Q$ is simply:
\[
\vec{F} = Q\vec{E}
\]
where $\vec{E}$ is the electric field at the charge. We can see that $\vec{E}$ 
also follows the law of superposition. Thus for a system of charges:
\[
\vec{E}(r) = \sum_{i = 1}^{N} \frac{1}{4\pi\epsilon_0}  \frac{q_i}{{\mathscr{r}_i}^2} \uvec{\mathscr{r}}_i
\]
where $\vec{\mathscr{r}}_i = \vec{r} - \vec{r}_i$, the vector from charge $q_i$ to $\vec{r}$.

For a continous charge distribution the sum becomes an integral:
\[
\vec{E}(r) = \int_{\mathcal{V}} \frac{1}{4\pi\epsilon_0} \frac{\rho \dd[3]{r}}{{\mathscr{r}}^2} \uvec{\mathscr{r}}
\]

\addstuff{Add field lines, quantisation/conservation of charge, how dq is fine even though quantisation.}

\subsection{Volume density for discrete charges}

A more general way \improvement{Motivate this properly} to look at discrete, line and surface charges is in terms of the delta function.
We can extend volume density to such places (which is much more physical) by using the delta function.
For discrete charges:
\[
\rho(\vec{r}) = \sum q_i \delta(\vec{r} - \vec{r}_i) = \sum q_i \delta(\vec{\mathscr{r}}_i).
\]

The extension to surface charges and line charges is similar, for instance for a spherical shell with 
surface charge $\sigma$ and radius $R$:
\[
\rho(\vec{r}) = \sigma \delta(\abs{\vec{r}} - R).
\]


\section{Gauss's Law}

Suppose a charge is enclosed by a surface--- for a concrete example consider a charged particle inside it. 
Then the amount of charge inside should be measure of the field lines, and thus the field. 

\[
\Phi = \int \vec{E} \vdot \uvec{n} \dd{a}  
\]

The force perpendicular to a surface charge, if the flux through it is $\Phi$ is:
\[
\vec{F} \vdot \uvec{n}  = \sigma \Phi 
\]

\subsection{Solid Angle}

\section{Symmetries}

\section{Scalar Potential}

Since the curl of $\vec{E}$ is 0, it follows that:
\[
\int_{S} (\curl{\vec{E}}) \vdot \uvec{n} \dd[2]{r} = \int_{\partial S} \vec{E} \vdot \dd{\vec{\ell}} \implies 
\int_{\partial S} \vec{E} \vdot \dd{\vec{\ell}} = 0.
\]

From this it follows that the line integral between any two points is independent of the path, so we define 
a fuction using it.

\begin{definition}
    We define the scalar potential $V$ as:
    \begin{equation}
        V(\vec{r}) = -\int_{\mathcal{O}}^\vec{r} \vec{E} \vdot \dd{\vec{\ell}} 
    \end{equation}
    where $\mathcal{O}$ is the origin.
\end{definition}

Thus,
\[
 V(\vec{b}) - V(\vec{a}) = -\int_{\vec{a}}^\vec{b} \vec{E} \vdot \dd{\vec{\ell}} 
\]
But from the gradient theorem,
\[
 V(\vec{b}) - V(\vec{a}) = \int_{\vec{a}}^\vec{b} \grad{V} \vdot \dd{\vec{\ell}} 
\]
So it follows that:
\begin{equation}
    \vec{E} = -\grad{V}.
\end{equation}

From the definition it follows that the potential due to a charge distribution is simply:
\[
V(\vec{r}) = \int_{\mathcal{V}} \frac{1}{4\pi\epsilon_0} \frac{\rho \dd[3]r}{\mathscr{r}^2} \vec{\mathscr{r}}
\]