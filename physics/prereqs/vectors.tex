\setlength{\chnumsep}{14em}
\chapter{Vectors}

\begin{overview}
In this chapter we'll give an introduction to vectors, and some results in
geometry that you may find useful. Vectors are not introduced rigorously here,
but we do use some facts from linear algebra.
\end{overview}

\section{Co-ordinate Free Vectors}

We first describe vectors as being co-ordinate free. And really, even
though we will introduce co-ordinates later, you should still
think of vectors as just existing out somewhere, with or without
co-ordinates.

\begin{marginfigure}
\incfig{freevector}
\caption{A directed line segment, the vector \(\vec{A}\).}
\end{marginfigure}

You can think of a \index{vector} vector as a \emph{directed line segment}.
Vectors are usually denoted using boldface, \(\vec{A}\). 

Because they're directed line segments, only their length matters and
we’re not concerned with the location of their end points. This means that a
vector can be freely translated, without altering it.

\begin{marginfigure}
\scalebox{1.8}{\incfig{translatedvector}}
\caption{Two identical vectors.}
\end{marginfigure}

The magnitude of the vector is its length, which we call it's \tpvocab{Vector}{norm}, and write
as \(\norm{\vec{A}}\) or simply \(A\). You may also see it being written as \(\abs{\vec{A}}\).

A unit vector, \(\uvec{\gamma}\) (“gamma hat”) is a vector whose magnitude is unity. We use the unit vector to
often denote the direction of a vector by multiplying the unit vector with its magnitude.
For instance, the unit vector that point is the direction of \(\vec{A}\), \(\uvec{A}\) can be calculated as,
\[
\uvec{A} = \frac{\vec{A}}{A}.
\]

Further, vectors have certain operations defined on them. The base operations
are \tpvocab{Vector}{Scalar Multiplication} and \tpvocab{Vector}{Vector Addition}.

\subsection{Scalar Multiplication}  

Scalar Multiplication refers to multiplying a vector by a \vocab{scalar}. A scalar
for our purposes refers to a real number. Consider a vector \(\vec{v}\).
Multiplying it by a scalar, \(\alpha \in \RR\) produces a vector 
\(\alpha\vec{v}\) parallel to the original vector and changes the magnitude of the vector so that
\(\norm{\alpha\vec{v}}\) is \(\abs{\alpha}\) times greater than \(\norm{\vec{v}}\), thus
\(\norm{\alpha \vec{v}} = \abs{\alpha}\norm{\vec{v}}\). \(\abs{\alpha}\) can 
be smaller or greater than \(1\) which accordingly increases/decreases the 
lenght of \(\vec{v}\).

\begin{marginfigure}
  \centering
  \scalebox{1.5}{\incfig{scalar multiplication}}
  \caption{Scalar multiplication of a vector \(\vec{A}\) by \(c > 1\) and \(-1\).}
\end{marginfigure}

If \(\alpha > 0\), the vector produced is in the same direction as the original vector. If
\(\alpha < 0\), the direction of the vector is reversed. 

\subsection{Vector Addition}

Adding two vectors, \(\vec{A}\) and \(\vec{B}\) produces another 
vector \(\vec{A + B}\). Geometrically, vector addition is done by placing
the tail of \(\vec{B}\) on the head of \(\vec{A}\), and then the vector joining
the tail of \(\vec{A}\) and head of \(\vec{B}\) is the vector \(\vec{A + B}\) as 
in \cref{subfig: vectoradd1}.

It can also be interpreted as the diagonal of the parallelogram made by placing
the tails of the two vectors together, and producing two sides parallel to them as 
show in \cref{subfig: vectoradd2}.

\begin{marginfigure}
  \centering
  \begin{subfigure}[t]{\marginparwidth}
    \scalebox{1.5}{\incfig{vectoradd1}}
  \caption{Adding vectors, parallelogram interpretation.}
  \label{subfig: vectoradd1}
  \end{subfigure}
  \begin{subfigure}[t]{\marginparwidth}
    \scalebox{1.2}{\incfig{vectoradd2}}
  \caption{Adding vectors, triangle interpretation.}
  \label{subfig: vectoradd2}
  \end{subfigure}
  \caption{Addition of two vectors, \(\vec{A}\) and \(\vec{B}\) produces another
  vector, \(\vec{A + B}\).}
\end{marginfigure}

Subtraction of vectors, \(\vec{A} - \vec{B}\) is equivalent to multiplying \(\vec{B}\) 
by \(-1\) and then adding it with \(\vec{A}\).

\section{Introducing Co-ordinates}

There are an infinite number of vectors in 3d space, of course,
but that makes it harder to deal with them. Since vectors 
are just lengths with some direction, could we perhaps assign them 
co-ordinates? 


Consider a plane, \(P\), and two non-parallel vectors in it,
\(\vec{v}\), \(\vec{w}\). Suppose we wanted 

\begin{example}
  For example \(\vec{e}_x = (1, 0)\) and \(\vec{e}_y = (0, 1)\) form a basis of \(\RR^2\).
\end{example}