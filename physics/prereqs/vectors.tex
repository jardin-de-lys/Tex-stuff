\setlength{\chnumsep}{14em}
\chapter{Vectors}

\begin{overview}
In this chapter we'll give an introduction to vectors, and some results in
geometry that you may find useful. Vectors are not introduced rigorously here,
but we do use some facts from linear algebra.
\end{overview}

\section{Co-ordinate Free Vectors}

We first describe vectors as being co-ordinate free. And really, even
though we will introduce co-ordinates later, you should still
think of vectors as just existing out somewhere, with our without
co-ordinates.

\begin{marginfigure}
\incfig{freevector}
\caption{A directed line segment, the vector \(\vec{A}\).}
\end{marginfigure}

You can think of a \index{vector} vector as a \emph{directed line segment}.
Vectors are usually denoted using boldface, \(\vec{A}\) or with an arrow over its name,
\(\ovec{A}\). We'll use the former, though the latter is often used when writing by hand.


Because they're directed line segments, only their length matters and
we’re not concerned with the location of their end points. This means that a
vector can be freely translated, without altering it.

\begin{marginfigure}
\scalebox{1.8}{\incfig{translatedvector}}
\caption{Two identical vectors.}
\end{marginfigure}

The magnitude of the vector is its length, which we call it's \tpvocab{Vector}{norm}, and write
as \(\norm{\vec{A}}\) or simply \(A\). You may also see it being written as \(\abs{\vec{A}}\).

A unit vector, \(\uvec{\gamma}\) (“gamma hat”) is a vector whose magnitude is unity. We use the unit vector to
often denote the direction of a vector by multiplying the unit vector with its magnitude.
For instance, the unit vector that point is the direction of \(\vec{A}\), \(\uvec{A}\) can be calculated as,
\[
\uvec{A} = \frac{\vec{A}}{A}.
\]
