%to set math fonts to use
\usepackage[cal=boondoxo,
calscaled=.94,
bb=dsfontserif,
frak=esstix,
frakscaled=.97,
scr=dutchcal]{mathalpha}
%ysefyk packages
\usepackage{xargs}
\usepackage{float}
\usepackage[nameinlink]{cleveref}
\crefname{exercise}{exercise}{questions}
\usepackage{graphicx, calc, subcaption}
\usepackage{emoji, enumitem}
\usepackage{physics}
%fix qty issue due to physics package
\AtBeginDocument{\RenewCommandCopy\qty\SI}
\usepackage{siunitx}
\usepackage{marginnote}
%fixing some physics issues
\renewcommand{\cross}{\times}
\renewcommand{\curl}[1]{\nabla\hspace{-0.1em} \times #1}

\usepackage[colorinlistoftodos,prependcaption,textsize=tiny]{todonotes}
\newcommandx{\unsure}[2][1=]{\todo[linecolor=red,backgroundcolor=red!25,bordercolor=red,#1]{#2}}
\newcommandx{\change}[2][1=]{\todo[linecolor=blue,backgroundcolor=blue!25,bordercolor=blue,#1]{#2}}
\newcommandx{\addstuff}[2][1=]{\todo[linecolor=OliveGreen,backgroundcolor=OliveGreen!25,bordercolor=OliveGreen,#1]{#2}}
\newcommandx{\improvement}[2][1=]{\todo[linecolor=Plum,backgroundcolor=Plum!25,bordercolor=Plum,#1]{#2}}
%
%For spacing around f after nabla
\usepackage{luacode}
\begin{luacode}
function nabla_f ( s )
   s = s:gsub ( "\\grad{f}" ,  "\\nabla\\mkern-2mu f" ) 
   s = s:gsub ( "\\partial%s*f", "\\partial\\mkern-4mu f" )
   return s
end
\end{luacode}
\directlua{luatexbase.add_to_callback("process_input_buffer", nabla_f , "nablaf" )}

%better typography
\usepackage[activate={true,nocompatibility},factor=1100,stretch=10,shrink=10]{microtype}
\microtypecontext{spacing=nonfrench}  % silence warning
\tcbuselibrary{skins}

%for testing font, math
\usepackage[math]{blindtext}

%use ell instead of l in math mode.
\mathcode`l="8000
\begingroup
\makeatletter
\lccode`\~=`\l
\DeclareMathSymbol{\lsb@l}{\mathalpha}{letters}{`l}
\lowercase{\gdef~{\ifnum\the\mathgroup=\m@ne \ell \else \lsb@l \fi}}%
\endgroup


\usepackage{centernot}
\usepackage{mathtools}
\usepackage{tikz-cd}
\usepackage{cancel}

%macros

\newcommand{\minusspace}{\hspace{-0.05em}}
\renewcommand{\vnabla}{\nabla}
\newcommand{\ovec}[1]{\overrightarrow{\mathrm{#1}}}
\newcommand{\increment}{\Delta}
\renewcommand{\emptyset}{\varnothing}
\renewcommand{\vec}[1]{\mathbf{\boldsymbol{#1}}}
\newcommand{\uvec}[1]{\vec{\hat{#1}}}
\newcommand{\conj}[1]{\overline{#1}}

\newcommand{\seq}[2]{\left\langle#1_1, #1_2, \dots, #1_{#2}\right\rangle}
\newcommand{\seqq}[1]{\left\langle#1\right\rangle}
\newcommand{\colvec}[1]{\langle #1\rangle}

\usepackage{braket}
\renewcommand*{\Set}[1]{\left\{#1\right\}}
\newcommand{\given}{\mid}
\newcommand{\cgiven}{:}

\newcommand{\avg}[1]{\langle#1\rangle}
\newcommand{\bavg}[1]{\overline{#1}}
\newcommand{\notiff}{%
  \mathrel{{\ooalign{\hidewidth$\not\phantom{"}$\hidewidth\cr$\iff$}}}}

\newcommand{\listvec}[2]{\left(#1_{1}, #1_{2}, \dots, #1_{#2}\right)}

\newcommand{\many}[2]{#1_{1}#1_{2}\cdots#1_{#2}}
\newcommand{\manys}[2]{\{#1_{1},#1_{2}\ldots,#1_{#2}\}}
\newcommand{\cbrt}[1]{\sqrt[3]{#1}}
\newcommand{\floor}[1]{\left\lfloor #1 \right\rfloor}
\newcommand{\ceil}[1]{\left\lceil #1 \right\rceil}
\newcommand{\mailto}[1]{\href{mailto:#1}{\texttt{#1}}}
\newcommand{\ol}{\overline}
\newcommand{\ul}{\underline}
\newcommand{\wt}{\widetilde}
\newcommand{\wh}{\widehat}
\newcommand{\eps}{\varepsilon}
\providecommand{\alert}{\vocab}
\providecommand{\half}{\frac{1}{2}}
\newcommand{\catname}{\mathsf}
\newcommand{\hrulebar}{
    \par\hspace{\fill}\rule{0.95\linewidth}{.7pt}\hspace{\fill}
    \par\nointerlineskip \vspace{\baselineskip}
}
\providecommand{\arc}[1]{\wideparen{#1}}

%For use in author command
\newcommand{\plusemail}[1]{\\ \normalfont \texttt{\mailto{#1}}}

%More commands and math operators
\DeclareMathOperator{\cis}{cis}
\DeclareMathOperator*{\lcm}{lcm}
\DeclareMathOperator*{\argmin}{arg min}
\DeclareMathOperator*{\argmax}{arg max}

%Convenient Environments
\newenvironment{soln}{\begin{proof}[Solution]}{\end{proof} \hrule}
\newenvironment{parlist}{\begin{inparaenum}[(i)]}{\end{inparaenum}}
\newenvironment{gobble}{\setbox\z@\vbox\bgroup}{\egroup}

%Inequalities
\newcommand{\cycsum}{\sum_{\mathrm{cyc}}}
\newcommand{\symsum}{\sum_{\mathrm{sym}}}
\newcommand{\cycprod}{\prod_{\mathrm{cyc}}}
\newcommand{\symprod}{\prod_{\mathrm{sym}}}

%From H113 "Introduction to Abstract Algebra" at UC Berkeley
\newcommand{\CC}{\mathbb C}
\newcommand{\FF}{\mathbb F}
\newcommand{\NN}{\mathbb N}
\newcommand{\NNO}{\mathbb N_{0}}
\newcommand{\ZZO}{\mathbb Z_{\ge 0}}
\newcommand{\RRO}{\mathbb R_{\ge 0}}
\newcommand{\QQ}{\mathbb Q}
\newcommand{\RR}{\mathbb R}
\newcommand{\ZZ}{\mathbb Z}

\newcommand{\charin}{\text{ char }}
\DeclareMathOperator{\sign}{sign}
\DeclareMathOperator{\Aut}{Aut}
\DeclareMathOperator{\Inn}{Inn}
\DeclareMathOperator{\Syl}{Syl}
\DeclareMathOperator{\Gal}{Gal}
\DeclareMathOperator{\GL}{GL} % General linear group
\DeclareMathOperator{\SL}{SL} % Special linear group
\DeclareMathOperator{\Vol}{Vol} % Special linear group

%From Kiran Kedlaya's "Geometry Unbound"
\newcommand{\dang}{\measuredangle} %% Directed angle
\newcommand{\ray}[1]{\overrightarrow{#1}}
\newcommand{\seg}[1]{\overline{#1}}

\newcommand{\ndiv}[1]{\dot{#1}}
\newcommand{\nddiv}[1]{\ddot{#1}}

%From M275 "Topology" at SJSU
\DeclareMathOperator{\id}{id}
\newcommand{\taking}[1]{\xrightarrow{#1}}
\newcommand{\inv}{^{-1}}

%From M170 "Introduction to Graph Theory" at SJSU
\DeclareMathOperator{\diam}{diam}
\DeclareMathOperator{\ord}{ord}
\newcommand{\defeq}{\overset{\mathrm{def}}{=}}

%From the USAMO .tex files
\newcommand{\ts}{\textsuperscript}
\newcommand{\dg}{^\circ}
\newcommand{\ii}{\item}

% From Math 55 and Math 145 at Harvard
\newenvironment{subproof}[1][Proof]{%
    \begin{proof}[#1] \renewcommand{\qedsymbol}{$\blacksquare$}}%
    {\end{proof}}

\newcommand{\liff}{\leftrightarrow}
\newcommand{\lthen}{\rightarrow}

\DeclareMathOperator{\Img}{Im} % Image
\DeclareMathOperator{\coker}{coker} % Cokernel
\DeclareMathOperator{\Coker}{Coker} % Cokernel
\DeclareMathOperator{\Ker}{Ker} % Kernel
\DeclareMathOperator{\Spec}{Spec} % spectrum
\DeclareMathOperator{\pr}{pr} % projection
\DeclareMathOperator{\ext}{ext} % extension
\DeclareMathOperator{\pred}{pred} % predecessor
\DeclareMathOperator{\dom}{dom} % domain
\DeclareMathOperator{\ran}{ran} % range
\DeclareMathOperator{\Hom}{Hom} % homomorphism
\DeclareMathOperator{\Mor}{Mor} % morphisms
\DeclareMathOperator{\End}{End} % endomorphism

% Things Lie
\newcommand{\kb}{\mathfrak b}
\newcommand{\kg}{\mathfrak g}
\newcommand{\kh}{\mathfrak h}
\newcommand{\kn}{\mathfrak n}
\newcommand{\ku}{\mathfrak u}
\newcommand{\kz}{\mathfrak z}
\DeclareMathOperator{\Ext}{Ext} % Ext functor
\DeclareMathOperator{\Tor}{Tor} % Tor functor
\newcommand{\SC}{\mathcal{S}}
\newcommand{\SCF}{\mathscr F}
\newcommand{\SCG}{\mathscr G}
\newcommand{\SCH}{\mathscr H}

% Mathfrak primes
\newcommand{\km}{\mathfrak m}
\newcommand{\kp}{\mathfrak p}
\newcommand{\kq}{\mathfrak q}


%aliases
%\renewcommand{\ge}{\geqslant}
%\renewcommand{\le}{\leqslant}
\renewcommand{\subset}{\subsetneq}
\newcommand{\auth}[1]{\emph{#1}}
\newcommand{\para}[1]{#1 \par}
\newcommand{\lpara}[1]{\par}  
\newcommand{\parbreak}{\smallskip}

%nice aliases
\newcommand{\union}{\cup}
\newcommand{\intersect}{\cap}
\newcommand{\im}{\mathrm{\iota}}

%nicer re and im
\renewcommand{\Re}{\operatorname{Re}}
\renewcommand{\Im}{\operatorname{Im}}

%for indexing
\newcommand{\vocab}[1]{{\itshape#1}\index{#1}}
\newcommand{\tpvocab}[2]{{\itshape#2}\index{#1!#2}}

%quicker macro
\newcommand{\dtp}{\dotproduct}

%spacing and euler's number
\newcommand{\e}{\,{\mathrm{e}}}
\renewcommand{\dd}[1][]{\mathop{}\!{d}\hspace{-0.1em}^{#1}\hspace{-0.11em}}
\renewcommand{\diffd}{\mathop{}\!{d}\hspace{-0.17em}}

%sidenotes
%monofont too large
\renewcommand{\UrlFont}{\ttfamily\scriptsize}

%easier basis vectors
\newcommand{\ihat}{\vec{i}}
\newcommand{\jhat}{\vec{j}}
\newcommand{\khat}{\vec{k}}
\newcommand{\that}{\vec{e}_{\theta}}
\newcommand{\rhat}{\vec{e}_{r}}

%i dont remember what this does
\usepackage{etoolbox}

\makeatletter
\patchcmd{\PEX@}{\ifdim\dp\Pbox@>\dp\z@}{\ifdim\ht\Pbox@>\dp\z@}{}{}
\makeatother

\makeatletter
\patchcmd{\PEX@}{\dp\Pbox@>\dp\z@}{\ht\Pbox@>\dp\z@}{}{}
\patchcmd{\SQEX@}{\dp\Sbox@>\dp0}{\ht\Sbox@>\dp0}{}{}
\makeatother